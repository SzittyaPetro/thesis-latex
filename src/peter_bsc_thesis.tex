%\documentclass[11pt,a4paper,oneside]{report}

\documentclass[11pt,a4paper,twoside,openright]{report}  % Duplex
%\usepackage{hyperref}             % Single-side
\input{include/packages}

%TODO Set the main variables
\newcommand{\vikszerzoVezeteknev}{Nyilas}
\newcommand{\vikszerzoKeresztnev}{Péter}

\newcommand{\vikkonzulensAMegszolitas}{dr.~}
\newcommand{\vikkonzulensAVezeteknev}{Hullám}
\newcommand{\vikkonzulensAKeresztnev}{Gábor}

\newcommand{\vikcim}{
Training and Interpretation of a Neural Network Model for Traffic Object Detection}
% Cím
\newcommand{\viktanszek}{\bmemit} % Tanszék
\newcommand{\vikdoktipus}{\bsc} % Dokumentum típusa (\bsc vagy \msc)
\newcommand{\vikmunkatipusat}{szakdolgozatot} % a "hallgató nyilatkozat" részhez: szakdolgozatot vagy diplomatervet

\input{include/tdk-variables}
\newcommand{\szerzoMeta}{\vikszerzoVezeteknev{} \vikszerzoKeresztnev} % egy szerző esetén
%\newcommand{\szerzoMeta}{\vikszerzoVezeteknev{} \vikszerzoKeresztnev, \tdkszerzoB} % két szerző esetén

%TODO Language configuration -- choose one
% Beállítások magyar nyelvű dolgozathoz
%\input{include/thesis-hu}
% Settings for English documents
\input{include/thesis-en}

\input{include/preamble}

%--------------------------------------------------------------------------------------
% Table of contents and the main text
%--------------------------------------------------------------------------------------
\begin{document}

\pagenumbering{gobble}

%TODO These includes define guidelines -- remove these
%~~~~~~~~~~~~~~~~~~~~~~~~~~~~~~~~~~~~~~~~~~~~~~~~~~~~~~~~~~~~~~~~~~~~~~~~~~~~~~~~~~~~~~
%\include{include/guideline}
%\include{include/project}
\includepdf{include/Kozlekedesi-objektumok-detekciojat-megvalosito-Feladatkiiras.pdf}

\selectthesislanguage

%~~~~~~~~~~~~~~~~~~~~~~~~~~~~~~~~~~~~~~~~~~~~~~~~~~~~~~~~~~~~~~~~~~~~~~~~~~~~~~~~~~~~~~
\include{include/titlepage}		   % Szakdolgozat/Diplomaterv címlap

\renewcommand{\thesection}{\arabic{section}}
\renewcommand{\thesubsection}{\thesection.\arabic{subsection}}
\renewcommand{\thesubsubsection}{\thesubsection.\arabic{subsubsection}}
% Table of Contents
%~~~~~~~~~~~~~~~~~~~~~~~~~~~~~~~~~~~~~~~~~~~~~~~~~~~~~~~~~~~~~~~~~~~~~~~~~~~~~~~~~~~~~~
\tableofcontents\cleardoublepage


% Declaration and Abstract
%~~~~~~~~~~~~~~~~~~~~~~~~~~~~~~~~~~~~~~~~~~~~~~~~~~~~~~~~~~~~~~~~~~~~~~~~~~~~~~~~~~~~~~
\include{include/declaration} %TODO Hallgatói nyilatkozat -- TDK és OTDK esetén törlendő!
\pagenumbering{roman}
\setcounter{page}{1}

\selecthungarian

%----------------------------------------------------------------------------
% Abstract in Hungarian
%----------------------------------------------------------------------------
\chapter*{Kivonat}\addcontentsline{toc}{chapter}{Kivonat}

Ez a dokumentum egy mesterséges intelligencia alapú rendszer fejlesztését mutatja be, amely képes automatikusan
felismerni és osztályozni különböző objektumokat képeken, illetve foglalkozik a neurális hálózat interpretálásával.
A rendszer a YOLOv8 modelt használja, amely egy mély neurális hálózat alapú objektumfelismerő algoritmus.
A dolgozat részletesen tárgyalja a modell betanításának és interpretálásának folyamatát, valamint az elért eredményeket
és levonja a következtetéseket.

\vfill
\selectenglish

%----------------------------------------------------------------------------
% Abstract in English
%----------------------------------------------------------------------------
\chapter*{Abstract}\addcontentsline{toc}{chapter}{Abstract}

This paper describes the development of an artificial intelligence-based system that can
recognise and classify different objects in images and interprets the neural network.
The system uses the YOLOv8 model, which is a deep neural network based object recognition algorithm.
The paper discusses in detail the process of training and interpreting the model and the results obtained
and draws conclusions.

\vfill
\cleardoublepage

\selectthesislanguage

\newcounter{romanPage}
\setcounter{romanPage}{\value{page}}
\stepcounter{romanPage}    %TODO Összefoglaló -- TDK és OTDK esetén nem kötelező


% The main part of the thesis
%~~~~~~~~~~~~~~~~~~~~~~~~~~~~~~~~~~~~~~~~~~~~~~~~~~~~~~~~~~~~~~~~~~~~~~~~~~~~~~~~~~~~~~
\pagenumbering{arabic}
\section{Introduction and Motivations}\label{sec:introduction}
%értelmezés
The accurate detection of objects that are relevant to the movement of vehicles is a critical task in the field of automotive and transportation systems.
Numerous scenarios can arise where different types of object detection is required.
The development of a highly sophisticated automated driving system requires the efficient gathering and processing of data from sensors located in the chassis to create a virtual model of the surrounding world.

%tervezés célja
The primary challenge of developing autonomous vehicles is the development of systems  to create a reliable and accurate perception of the vehicle's environment.
One of the most widely used solutions is the use of camera sensors as they are able to
provide rich visual data that can be processed rapidly and can give the most information by itself. The goal of this project was to develop a software capable of fully fulfilling this need.

%értelmezés
Nevertheless, the most significant limitation of these models is that they are frequently regarded as black boxes, lacking a straightforward correlation between input and output.
This lack of transparency in how models arrive at conclusions has led to the development of numerous model interpretation techniques, as evidenced by the work of a number of different researchers~\cite{LIANG2021168}.
These techniques vary in their approach, but they are all aiming to facilitate an insight into the decision-making process of the model.

%feladat indokoltsága
The complexity of interpreting models in these domains arises from the intricate nature of the data and the sheer size of models used.
This complexity makes it challenging to trace the decision-making process, yet it is essential for identifying potential biases,
improving model performance, and gaining the trust of users and regulatory bodies.
Therefore, with the will of enhancing the interpretability of this project's model is vital for maintaining safety and ethical standards, given the nature of the application.

%tervezés célja
The secondary goal of this project was to try and implement different interpretation methods,
aimed to interpret the model and try to create explanations locally about the output our model produces.


\subsection{Structure}\label{subsec:Structure} % felépítés rövid összefoglalása

The main parts of my thesis are as follows:

\begin{itemize}
    \item Literature study about Image processing Deep Convolutional Neural Networks
    \item Yolo architecture and general overview on training and infrastructure.
    \item Cityscapes Dataset and data processing and representation.
    \item Evaluation of model performance
    \item Literature review about model-agnostic and  model-dependent interpretation methods
    \item Implementation, use and evaluation of EigenCAM, LIME and SHAP.
    \item To analyze the results and draw conclusions on the effectiveness of the model and the XAI solutions for traffic object detection.
\end{itemize}
\section{Image Processing Deep Neural Networks}\label{sec:image-processing-deep-neural-networks}
\section{Image Processing Deep Neural Networks}\label{sec:image-processing-deep-neural-networks}

%Checked
\subsection{Overview of Convolutional Neural Networks}\label{subsec:convolutional-neural-networks}

Convolutional neural networks (CNNs) constitute a class of deep learning models that are commonly used for computer vision tasks
including image classification, object detection, and segmentation.
Given the ability of convolutional neural networks (CNNs) to learn spatial hierarchies of features and their various types of data representations
(bounding box, segmentation mask, etc.), they are made ideal for the task of traffic object detection.
Many recent advances  in computer vision have been made possible by CNNs, including the development of models such as YOLO, Faster R-CNN, and ResNet.
A dozen of these models have been developed, each with its own unique architecture and capabilities.
However, they all share the same fundamental principles gathered recently by Jiuxiang Gu in his 2018 work,
\("\)Recent advances in convolutional neural networks\("\)~\cite{GU2018354}.

The designation of the class is derived from the utilisation of convolutional layers, which apply filters to the input
data in order to extract features, outside convolutional layers, CNNs also comprise pooling layers, fully
connected layers.

These models are employed in a multitude of applications, including autonomous vehicles, surveillance systems, and medical imaging.
As these applications align with the project's objectives, these types of neural networks are selected for the task of traffic object detection.

The subsequent paragraphs will provide an overview of the essential components of convolutional neural networks (CNNs).

\paragraph{Convolutional layers}\label{par:convolutional-layers}

Convolutional layers  represent the fundamental building blocks of convolutional neural networks (CNNs).
Convolution operations are applied to the input data using filters (or in other word kernels) that process the input image.
This process enables the model to identify local patterns such as  edges, textures, and shapes.
Each convolutional layer generates a set of feature maps,
which indicate the presence of different types of these aforementioned features in the input.
These feature maps are then passed to the subsequent layers for further processing, such as pooling or classification,
to refine the information coded into the image.
During the training phase, the parameters of the kernels are learned (their values are adjusted),trough backpropagation,
allowing the network to enhance feature extraction for specific tasks.


\begin{figure}[h]
\centering
\includegraphics[width=.50\textwidth]{figures/convolution}
\caption{Convolutional layer operation~\cite{article}}
\label{fig:convolution}
\end{figure}

\paragraph{Sampling layers}\label{par:sampling-layers}

The objective of sampling layers is to reduce the dimensionality of the data set while ensuring the preservation of the
input data's essential features.
Sampling can entail the application of techniques such as subsampling or strided convolutions,
whereby a specific stride is applied to the convolution operation in order to down-sample the feature maps.

\paragraph{Pooling layers}\label{par:pooling-layers}

The objective of pooling layers is to reduce the spatial dimensions of feature maps given by the convolutional layers,
thereby decreasing the number of parameters and computations in the network.
Moreover, this process enhances the model's resilience to minor
translations in the input data by mitigating the effects of potential outliers.
The most prevalent forms of pooling are max pooling and average pooling.
Max pooling selects the maximum value from a specified window, whereas average pooling computes the average.
By down-sampling the feature maps, pooling layers assist in maintaining the most important features while discarding less
critical information from the picture, also efficient in reducing the computational load.



\begin{figure}[h]
\centering
\includegraphics[width=.50\textwidth]{figures/pooling}
\caption{Pooling layer operation~\cite{article}}
\label{fig:pooling}
\end{figure}

\paragraph{Fully connected layers}\label{par:fully-connected-layers}

Fully connected layers, also referred to as dense layers, are typically located at
the conclusion(HEAD) of convolutional neural network (CNN) architectures.
In these layers, each neuron is connected to every neuron in the preceding layer.
This structure enables the model to integrate information from all features and make final predictions.
Fully connected layers are particularly useful in classification tasks,
where they compute the output probabilities for each class based on the features extracted by the preceding layers.
Regularisation techniques, such as dropout, are frequently employed in these layers to prevent overfitting.
\begin{figure}[ht]
\centering
\includegraphics[width=0.50\textwidth]{figures/fully connected}
\caption{Fully Connected Layer semantics~\cite{article}}
\label{fig:fullconn}
\end{figure}

\subsection{Activation Functions}\label{subsec:activation-functions}

Activation Functions are utilised in CNNs(and other Neural networks) to introduce non-linearity into the model.
They assist the model in learning complex patterns and making accurate predictions.
 In light of the work conducted by Siddharth Sharma and Simone Sharma regarding  activation functions~\cite{sharma2017activation}
the most prevalent  activation functions are ReLU, Sigmoid, and Tanh.
Although alternative activation functions, such as BSF and ELU, could be employed, they are not utilised by the examined model (Yolov8).
Consequently, a detailed discussion of these functions will not be provided.
The following paragraphs  will discuss the ReLU, Sigmoid, and Tanh activation functions based on the work of Siddharth and Simone Sharma\cite{sharma2017activation}.


\begin{figure}[h!]
    \centering


    \begin{subfigure}[b]{0.5\textwidth}
        \centering
        \includegraphics[width=\textwidth]{figures/sigmoid}
        \caption{Sigmoid activation function}
        \label{fig:sigmoid}
    \end{subfigure}
    \hfill
    \begin{subfigure}[b]{0.45\textwidth}
        \centering
        \includegraphics[width=\textwidth]{figures/tanh}
        \caption{Tanh activation function}
        \label{fig:tanh}
    \end{subfigure}
    \hfill
    \begin{subfigure}[b]{0.45\textwidth}
        \centering
        \includegraphics[width=\textwidth]{figures/relu}
        \caption{ReLU activation function}
        \label{fig:relu}
    \end{subfigure}

    \caption{Activation functions: Sigmoid,Tanh and ReLU}
    \label{fig:activation_functions}
\end{figure}


\paragraph{ReLU}\label{par:relu}
The Rectified Linear Unit (ReLU) is one of the most common and easy to understand activation functions in convolutional neural networks (CNNs).
It is defined as \( f(x) = \max(0, x) \) in the works of Siddharth Sharma~\cite{sharma2017activation}.
This function introduces non-linearity while maintaining a simple and efficient computation.
The ReLU function helps mitigate the vanishing gradient issue, allowing models to learn faster and perform better.
However, be susceptible to the phenomenon known as the \("\)dying ReLU\("\) problem, where neurons become inactive and only output zeros,
particularly during the training of deep networks.



\paragraph{Sigmoid}\label{par:sigmoid}
The Sigmoid function translates input values to a range between 0 and 1,
rendering it useful for binary classification problems.
It is defined as \( f(x) = \frac{1}{1 + e^{-x}} \) in the works of Siddharth Sharma~\cite{sharma2017activation}.
While the Sigmoid function provides smooth gradients,
it is prone to the vanishing gradient problem,
especially for large positive or negative input values.
This can slow down the training of deep networks,
which is why it is often replaced by other activation
functions in hidden layers,
though it still finds use in the output layer for binary classification tasks.



\paragraph{Tanh}\label{par:tanh}
The hyperbolic tangent (Tanh) function is another activation function that maps input values to a range between -1 and 1.
It is defined as  \( f(x) = \frac{e^x - e^{-x}}{e^x + e^{-x}} \)in the works of Siddharth Sharma~\cite{sharma2017activation}.
The Tanh function is zero-centered and centricaly symmetrical, facilitates the centring of the data and may result in accelerated convergence during training.
Nevertheless, it continues to exhibit the vanishing gradient problem for large input values, though to a lesser extent than the Sigmoid function.



\subsection{Introduction to YOLO}\label{subsec:introduction-to-yolo}

The YOLO (You Only Look Once) model is a relatively recent object detection system that is straightforward to utilise and comprehend, while also benefiting from a thriving community of users and developers.
It is based on the YOLO algorithm , which is a real-time object detection algorithm developed by Joseph Redmon and Ali Farhadi in 2015\cite{redmon2016lookonceunifiedrealtime}.

Unlike traditional object detection methods that apply classifiers to various sections of an image,
YOLO approaches the problem as a single regression problem.
The algorithm divides the image into a grid and  predicts bounding boxes and class probabilities parallel for each grid cell,
thereby enabling the detection of multiple types and instances of objects in a single run.
This architecture not only enhances speed but also improves detection accuracy by reducing the number of false positives.

The YOLO model has evolved through multiple versions, with improvements in both performance and varying capability.
Subsequent versions of the model, including YOLOv2, YOLOv3, and the latest YOLOv5 and YOLOv7, as well as Yolov8 (with Yolov10 and 11 forthcoming),
have introduced advancements in network architecture,
feature extraction, and training techniques.
These developments have rendered YOLO a suitable candidate for a plethora of applications,
including autonomous vehicles, as evidenced by the author's own observations, surveillance systems, and real-time video analysis.

A further noteworthy attribute of the Yolo-type neural networks is their scalability and versatility in terms of model architecture.
These models are available in a range of sizes (\textit{nano, small, medium, large, extralarge}) and with a variety of detection types (\textit{semantic segmentation, bounding boxes,
oriented bounding boxes, instance segmentation}), which can be deployed in diverse scenarios.




\begin{figure}[ht]
\includegraphics[width=1.0\textwidth]{figures/table1}
\caption{Different sizes of the YOLOv8~\cite{githubGitHubUltralyticsultralytics}}
\label{fig:tableofsizes}
\end{figure}

%táblázat a Yolo architektúrákról és méreteikről hasznos lehet ide

\subsection{Model architecture}\label{subsec:architecture}
This network, like many others in the CNN family, has a distinctive architectural configuration that
enables the execution of intricate tasks such as object detection and classification.
The system is constituted of three distinct and discrete components,
each comprising a unique set of layers that perform specific and separate functions.

\begin{figure}[ht]
\includegraphics[width=1.0\textwidth]{figures/Detailed-illustration-of-YOLOv8-model-architecture-The-Backbone-Neck-and-Head-are-the}
\caption{The architecture of the YOLOv8 model~\cite{FractureDetection2024}}
\label{fig:architecture}
\end{figure}

\paragraph{Backbone}\label{par:backbone}
The YOLOv8 model is based on convolutional neural networks (CNNs) that have been specifically designed to capture essential features from input images.
It comprises multiple layers of convolutional operations (called Conv and a complex layer called C2F)
that extract progressively more sophisticated representations.
This structure emphasises both depth and computational efficiency,
enabling the model to discern subtle details while maintaining rapid processing capabilities.
Innovations such as skip connections and normalization techniques are employed to enhance the learning dynamics and improve the model's robustness to variations in input conditions.
%kép a backbone-ról

\paragraph{Neck}\label{par:neck}
In the YOLOv8 architectural design, the neck serves as an intermediary between the backbone and the head.
The primary function of the neck is to consolidate features from the various levels of the backbone,
thereby enhancing the model's ability to detect objects across different scales.
By employing strategies such as feature fusion or pyramid pooling,
the neck effectively integrates both coarse and fine features,
enabling the model to better handle overlapping objects and diverse scene contexts.
This feature aggregation is crucial for optimising the detection performance,
as it allows the model to harness a comprehensive range of information.

%kép a neckről


\paragraph{Head}\label{par:head}
The head of the YOLOv8 model is responsible for generating the final outputs,
which are based on the features that have been processed through the neck.
The final stage of the YOLOv8 model translates the aggregated feature maps into bounding box predictions and associated class scores for each detected object.
This section typically employs a combination of convolutional and fully connected layers to refine these outputs,
ensuring they are accurate and meaningful.
Additionally, the head may implement techniques such as adaptive anchors or
confidence scoring to enhance the localisation and classification accuracy.
By effectively synthesising the rich feature information, the head enables YOLOv8 to achieve high-performance object detection suitable for a wide array of applications.

%kép a headről
\subsection{Training}\label{subsec:training}
The training process~\cite{redmon2016lookonceunifiedrealtime} for the YOLOv8 model involves feeding its input with a large dataset of labeled images.
The model learns to identify objects by minimizing the difference between its predictions and the actual labels trough multiple iterations:
After a training cycle ,if the default settings are kept, a validation function (so called val) is run,
to determine more information on the model's performance on pictures that are not present in the training set.
The training process is iterative, with the model adjusting its parameters to improve its predictions over time.
This process is computationally intensive and requires access to powerful hardware,
the best fitting hardware for this purpose, that can be found in an ordinary PC, is the GPU\@.

The training process involves several key steps, including data preprocessing, model initialization,
loss calculation, and parameter optimization.
The model is trained using a technique called backpropagation,which involves adjusting the model's weights based on
the error between its predictions and the ground truth labels.

The Yolov8 model uses a number of different loss functions,according to the original paper~\cite{redmon2016lookonceunifiedrealtime}
, loss functions to measure the difference between its predictions and the actual labels in different ways:



\begin{itemize}
\item \textbf{CIoU (Complete Intersection over Union) loss}:
\[
\text{CIoU} = 1 - \left( \text{IoU} - \frac{\rho^2(\mathbf{b}, \mathbf{b}^g)}{c^2} - \alpha v \right) ~\cite{li2020generalizedfocallosslearning}
\]
where \(\rho\) is the Euclidean distance between the center points of the predicted box \(\mathbf{b}\) and
 the ground truth box \(\mathbf{b}^g\), \(c\) is the diagonal length of the smallest enclosing box covering
 the two boxes, \(\alpha\) is a positive trade-off parameter, and \(v\) measures the consistency of aspect
 ratio.
 It is more sensible to localisation accuracy.

\item \textbf{DFL (Distribution Focal Loss)}:
\[
\text{DFL} = -\sum_{i=1}^{N} p_i \log(q_i)~\cite{DBLP:journals/corr/abs-1911-08287}
\]
where \(p_i\) is the true probability distribution and \(q_i\) is the predicted probability distribution.
It is sensible to the accuracy classification.
\item \textbf{BCE (
binary cross-entropy for classification loss
)}:
\[
\text{BCE} = -\sum_{i=1}^{N} y_i \log(p_i) + (1 - y_i) \log(1 - p_i)~\cite{ruby2020binary}
\]
where \(y_i\) is the true label and \(p_i\) is the predicted probability.

\end{itemize}



I opted for training the model on my local machine, using a single GPU, while it provided me with sufficient performance
, the training time was significantly longer than it would have been on a more powerful cloud machine.
To reduce the chance of overfitting, the training dataset is typically divided into training and validation sets,
with the latter used to evaluate the model's performance on unseen data.
The trainings batch size where set to 3, which is not common, but it was necessary to fit the model on the GPU,
to optimise training time.

\subsection{Training and evaluation with the help of MLOps solutions}\label{subsec:training-and-evaluation-with-the-help-of-mlops-solutions}
In order to monitor the performance of the model and to facilitate the visualisation of the learning process,
as well as to organise the experiments, an online machine learning monitoring solution, Comet.ml, has been selected.
This tool is capable of tracking the training process in real time and of visualising the properties of the model on
a user-friendly dashboard, which can be accessed from any device with an internet connection.
The Comet.ml platform also provides a range of features that can be used to compare different experiments,
such as hyperparameter tuning, model versioning, and collaboration tools, while also offering a comprehensive
set of APIs for integration with other tools and platforms.

\begin{figure}[ht]
\centering
\includegraphics[width=1.0\textwidth]{figures/model}
\caption{Comet.ml dashboard about our the moodel}
\label{fig:comet}
\end{figure}

Figure~\ref{fig:comet} depicts the Comet.ml dashboard, which offers a comprehensive representation of the training process,
encompassing loss and accuracy metrics, the learning rate, and the model's performance on the validation set.
This information is vital for evaluating the model's performance and for identifying potential issues that may arise during training.
Furthermore, the Comet.ml platform enables the comparison of different experiments,
which can be beneficial for fine-tuning the model's hyperparameters and for improving its performance.
Which I did try out, for comparing the performances of the same model on the same dataset,
but with different epochs and batch sizes to see and find a good balance between training time and performance.

\section[Chosen model, training and data management]{The Model, its training and data management} \label{sec:model-training-data}

\subsection{Introduction to YOLO}\label{subsec:introduction-to-yolo}

The YOLO (You Only Look Once) model is a cutting-edge object detection system that has gained a reputation for its speed and accuracy.
It is based on the YOLO algorithm , which is a real-time object detection algorithm developed by Joseph Redmon and Ali Farhadi in 2015\cite{redmon2016lookonceunifiedrealtime}.

Unlike traditional object detection methods that apply classifiers to various sections of an image,
YOLO approaches the problem as a single regression problem.
It divides the image into a grid and simultaneously predicts bounding boxes and class probabilities for each grid cell,
allowing it to detect multiple objects in a single pass.
This architecture not only enhances speed but also improves detection accuracy by reducing the number of false positives.

The YOLO model has evolved through multiple versions, with improvements in both performance and varying capability.
Subsequent versions of the model, including YOLOv2, YOLOv3, and the latest YOLOv5 and YOLOv7, as well as Yolov8 (with Yolov10 and 11 forthcoming),
have introduced advancements in network architecture,
feature extraction, and training techniques.
These developments have rendered YOLO a suitable candidate for a plethora of applications,
including autonomous vehicles, as evidenced by the author's own observations, surveillance systems, and real-time video analysis.

A further noteworthy attribute of the Yolo-type neural networks is their scalability and versatility in terms of model architecture.
These models are available in a range of sizes (\textit{nano, small, medium, large, extralarge}) and with a variety of detection types (\textit{semantic segmentation, bounding boxes,
oriented bounding boxes, instance segmentation}), which can be deployed in diverse scenarios.

For my work, I chose the~\textbf{Yolov8m} configuration, which stands for \textbf{Yolo version 8 medium}.
My choice was made on the bases of previous experience with this model architecture and the popularity of its applications.

%táblázat a Yolo architektúrákról és méreteikről hasznos lehet ide

\subsection{Model architecture}\label{subsec:architecture}
This network, like many others in the CNN family, has a distinctive architectural configuration that
enables the execution of intricate tasks such as object detection and classification.
The system is constituted of three distinct and discrete components,
each comprising a unique set of layers that perform specific and separate functions.

\paragraph{Backbone}\label{par:backbone}
The YOLOv8 model is based on convolutional neural networks (CNNs) that have been specifically designed to capture essential features from input images.
It comprises multiple layers of convolutional operations (called Conv and a complex layer called C2F)
that extract progressively more sophisticated representations.
This structure emphasises both depth and computational efficiency,
enabling the model to discern subtle details while maintaining rapid processing capabilities.
Innovations such as skip connections and normalization techniques are employed to enhance the learning dynamics and improve the model's robustness to variations in input conditions.
%kép a backbone-ról

\paragraph{Neck}\label{par:neck}
In the YOLOv8 architectural design, the neck serves as an intermediary between the backbone and the head.
The primary function of the neck is to consolidate features from the various levels of the backbone,
thereby enhancing the model's ability to detect objects across different scales.
By employing strategies such as feature fusion or pyramid pooling,
the neck effectively integrates both coarse and fine features,
enabling the model to better handle overlapping objects and diverse scene contexts.
This feature aggregation is crucial for optimising the detection performance,
as it allows the model to harness a comprehensive range of information.

%kép a neckről


\paragraph{Head}\label{par:head}
The head of the YOLOv8 model is responsible for generating the final outputs,
which are based on the features that have been processed through the neck.
The final stage of the YOLOv8 model translates the aggregated feature maps into bounding box predictions and associated class scores for each detected object.
This section typically employs a combination of convolutional and fully connected layers to refine these outputs,
ensuring they are accurate and meaningful.
Additionally, the head may implement techniques such as adaptive anchors or
confidence scoring to enhance the localisation and classification accuracy.
By effectively synthesising the rich feature information, the head enables YOLOv8 to achieve high-performance object detection suitable for a wide array of applications.

%kép a headről

\subsection{Training}\label{subsec:training}
The training process~\cite{redmon2016lookonceunifiedrealtime} for the YOLOv8 model involves feeding its input with a large dataset of labeled images.
The model learns to identify objects by minimizing the difference between its predictions and the actual labels trough multiple iterations:
After a training cycle ,if the default settings are kept, a validation function (so called val) is run,
to determine more information on the model's performance on pictures that are not present in the training set.
The training process is iterative, with the model adjusting its parameters to improve its predictions over time.
This process is computationally intensive and requires access to powerful hardware, such as GPUs.

The training process involves several key steps, including data preprocessing, model initialization,
loss calculation, and parameter optimization.
The model is trained using a technique called backpropagation,which involves adjusting the model's weights based on
the error between its predictions and the ground truth labels.
The Yolov8 model uses a number of different loss functions to measure the difference between its predictions and the actual labels:

\begin{itemize}
\item CIoU(Complete Intersection over Union) loss for bounding box regression to improve localisation accuracy.
\item DFL loss (Distribution Focal Loss) It helps the model to more accurate classification.
\item VFL loss (Varifocal Loss) It's designed to address imbalances and uncertainties in classification tasks.
\end{itemize}


I opted for training the model on my local machine, using a single GPU, while it provided me with sufficient performance
, the training time was significantly longer than it would have been on a more powerful cloud machine.
To reduce the chance of overfitting, the training dataset is typically divided into training and validation sets,
with the latter used to evaluate the model's performance on unseen data.
The trainings batch size where set to 3, which is not common, but it was necessary to fit the model on the GPU,
to optimise training time.

\subsection{Training and evaluation with the help of MLOps solutions}\label{subsec:training-and-evaluation-with-the-help-of-mlops-solutions}
In order to monitor the performance of the model and to facilitate the visualisation of the learning process,
as well as to organise the experiments, an online machine learning monitoring solution, Comet.ml, has been selected.
This tool is capable of tracking the training process in real time and of visualising the properties of the model on
a user-friendly dashboard, which can be accessed from any device with an internet connection.
The Comet.ml platform also provides a range of features that can be used to compare different experiments,
such as hyperparameter tuning, model versioning, and collaboration tools, while also offering a comprehensive
set of APIs for integration with other tools and platforms.

I used it to monitor the training process of the YOLOv8 model and to evaluate its performance on the Cityscapes dataset,
as well as to compare the results with other versions of the model configurations.


\subsection{Dataset and formats}\label{subsec:dataset-and-formats}

\paragraph{Cityscapes}\label{par:cityscapes}
The Cityscapes dataset is a large-scale dataset~\cite{Cordts2016Cityscapes} used for training and evaluating object detection models.
It contains high-resolution images of urban scenes, with detailed annotations for various objects such as cars,
pedestrians, and traffic signs.
The dataset is widely used in the field of computer vision for tasks such as semantic segmentation and object detection.

I chose the gtFine dataset, consisting precisely labelled segmentation masks.
Based on their work~\cite{Cordts2016Cityscapes} the set consists of 5000 images, with a resolution of 1024x2048 pixels, and annotations for 30 classes of objects.
It is divided into three subsets: training, validation, and test, with 2975, 500, and 1525 images, respectively.
The dataset is annotated using pixel-level segmentation masks, which provide detailed information about the
location and shape of objects in the scene.
However for the purpose of this work, the annotations were converted into bounding box format,
which is more suitable for object detection tasks.

\paragraph{Format conversion and datatypes}\label{par:formatconversion}
The images and annotations are converted into a format that the YOLOv8 model can process, which is
a text file containing the image path and the coordinates of the bounding boxes for each object in the image.
This process uses a custom script that reads the annotations from the Cityscapes dataset and converts the labels
into labels whose classes are filtered and transformed into the grouping I chose for this project.
The classes were grouped into five categories:
\begin{itemize}
    \item \textb{small vehicle}\textit{(usually cars, which are for personal use)},
    \item \textb{large vehicle}\textit{(busses, trucks and other large non personal vehicles)},
    \item \textb{two wheelers}\textit{(bicycles nad motorcycles)},
    \item \textb{On-rails}\textit{(trains and trams though the smaller Fine dataset didn't include any)}
    \item and \textb{person} \textit{(pedestrian, and rider)}.
\end{itemize}

The script also converts the semantic segmentation masks into bounding boxes, trough finding the most extreme points and
create a bounding box around them.
This converted output is then saved in a text file, which is used as input for the YOLOv8 model.

At the end the structure of the data is as follows:
\begin{itemize}
    \item \textb{Root(gtFine)}: The root directory of the Cityscapes dataset, which contains the images and annotations.
    \begin{itemize}
        \item \textb{labels}: Folder containing the images in the dataset, broken down into training, validation, and test sets
    and further divided into subfolders based on the city where the images were captured.
        \item \textb{labels}: The class label for each object in the image.
        \item \textb{train.txt}: The training set, which contains the paths to the training images and their corresponding annotations.
        \item \textb{val.txt}: The validation set, which contains the paths to the validation images and their corresponding annotations.
        \item \textb{test.txt}: The test set, which contains the paths to the test images and their corresponding annotations.
    \end{itemize}
    \item \textb{Descriptors}: The folder containing the class labels and their corresponding indices, as well as
    path set descriptor txt-s.
    It's used by the model to determine the classes, their indices and the paths to the images.
\end{itemize}

\subsection{Model evaluation}\label{subsec:model-evaluation}
The evaluation of the YOLOv8 model is performed using a set of metrics that measure its performance on the Cityscapes dataset.
These metrics include precision, recall,mAP, and IoU, which are commonly used in object detection tasks.


%----------------------------------------------------------------------------------------%

%! Author = NyPeter
%! Date = 2024. 10. 13.

\section{Model Interpretation}\label{sec:model-interpretation}

\subsection{Importance of Interpretation}\label{subsec:importance-of-interpretation}

Interpreting machine learning models is essential for understanding their decision-making processes.
It helps in identifying biases, improving model performance by providing information helping us the augment data more efficiently, and building trust with users.
Interpretation techniques provide insights into how image processing models make predictions and highlight the most influential features.

Interpretation of machine learning algorithms is a fairly new field, but it has gained significant attention
in recent years due to the increasing complexity of models and the need for transparency and accountability.
This need comes form industrial and judicial actors, who require explanations for the decisions made by models,
especially in safety-critical applications such as autonomous vehicles, healthcare, and finance.
In the last couple of years, the field of model interpretation has seen significant advancements, each brought
us closer to understanding the inner workings of machine learning models, and try to put reason and logic behind
otherwise black-box models.

Furthermore, based on these advancements the aforementioned actors are more willing to adopt laws,
standards and regulations that require models to be interpretable,
if they are to be used in non-research areas.

Building on the aforementioned state of the machine learning world, I felt it was important to try
and create a fairly coherent interpretation of the model to meet today's industry standards.

\subsection{Introduction to Interpretation methods}\label{subsec:introduction-to-interpretation-methods}

Model interpretation is a critical aspect of machine learning that aims to explain the decision-making process of models. % Ismétlés
Interpretable models are essential for building trust with users, identifying biases, and with this information,
developers have a deeper understanding that can help them improve model performance.
In this section, there will be a discussion of the importance of model interpretation and a review of different methods of interpretation.

In their survey~\cite{LIANG2021168}, Liang et al. highlight the significance of model interpretation techniques in providing insights from the model itself, and the output of the model, without knowing the inner state of the model.

Therefore they sorted these techniques into two categories:
\begin{enumerate}
    \item data driven or more commonly as model-agnostic methods
    \item model driven or model-specific methods
\end{enumerate}

The following sections will provide an overview of the two categories.
In the event that one is used, an analysis of the algorithms will be presented in light of the implementation used in this project.
This will include a comparison and contrast of their respective strengths and weaknesses.



\subsection{Model-Agnostic Methods}\label{subsec:model-agnostic-methods}
Based on the work of  Liang~\cite{LIANG2021168}, model-agnostic interpretation methods can be divided into three categories:
\begin{enumerate}
    \item Perturbation-based interpretation and a Game theoretic approach
    \item Adversarial-based interpretation
    \item Concept-based interpretation
\end{enumerate}


As discussed in the work of Liang~\cite{LIANG2021168}, these are designed to explain model predictions without relying
on the internal structure of the model hence their name, making them applicable to a wide range of models.
This insight may be more applicable to local examples, as the approach only provides explanations for the given data and not globally for the model itself.
For image processing purposes, these methods are particularly easy to understand and implement.

Both the game-theoretic and perturbation-based interpretation methods involve altering the input data
and observing the resulting changes in the predictions of the model on the modified image, while Adversarial-based interpretation methods focus on generating small perturbations or adversarial examples that can significantly change the predictions, highlighting the model's vulnerabilities.

On the other hand, concept-based interpretation methods aim to link model behavior to high-level human-understandable concepts.

Regardless, I will discuss the different types of data-driven interpretation methods in the following paragraphs, each on the basis of Liang's work~\cite{LIANG2021168}

\subsubsection{Perturbation-based Interpretation}\label{subsubsec:pertubation-based-interpretation}

Perturbation-based interpretation methods are model-agnostic techniques that explain model predictions by perturbing the input data.
These methods generate perturbed samples by introducing noise (or masking portions of the image) to the input image and observing the alterations in the model's predictions.
This masking is the main principle behind these methods.

The most common type of masking is occlusion, where parts of the image are covered to determine their importance for the model's output.
It can be interpreted as a form of feature selection, where the model's output is evaluated based on the presence or absence of specific features.

In essence, perturbation-based methods can be conceptualised as an optimisation problem.
Given an input vector \( x \), a model function \( f(x) \), and a predicted vector \( y \),
the objective is to identify which components of \( x \) contribute to the generation of the prediction. To accomplish this, the input features are systematically altered by introducing perturbations \( \delta x \)).
This approach allows us to infer the importance of specific features by quantifying the extent to which the predicted vector \( f(x + \delta x) \).
This approach allows us to infer the importance of specific features by quantifying the extent to which the predicted vector y is affected by the introduction of these perturbations.



Perturbation-based methods are focusing on \("local\) \(interpretability"\), meaning they explain specific predictions rather than providing an overarching understanding of the model’s behaviour.
This local explanation is vastly inferior to global interpretability.
Regardless, sometimes it can be extended, to provide broader insights into the models logic.

It works in a similar way to the human visual system and the way we perceive the visual world,
so that by obscuring different parts of an object, we can significantly alter our own eye's perception of the object.
By analysing the effect of perturbations on the model's output, these methods identify the most important features for a given prediction.

As a prominent implementation of this method is LIME (Local Interpretable Model-agnostic Explanations), which will be discussed in a following section section~\ref{subsec:methods-for-model-interpretation}, it was selected
for use in the project due to its simplicity and effectiveness in interpreting the model's predictions.


There are multiple types of perturbations:
\begin{itemize}
    \item \textbf{Noise insertion}: adding some type of noise (Gaussian or random) to parts of the input data.
    which can reveal the sensitivity of the model to small variations.
    \item \textbf{Blurring/Pixel Modification}: when working with image data, perturbations can be created by blurring parts of an image or modifying pixel intensities, which helps to understand how the clarity or sharpness of particular regions affects the prediction.
    \item \textbf{Feature Deletion for structured data}: remove or zeroing specific features, determining how significant that feature really is to the prediction.
    \item \textbf{Text Data Perturbations}: for text data, this involves removing words, phrases, characters, but it can also mean swapping tokens with synonyms to gauge their importance to the output.
\end{itemize}

Given the project's domain, I only used perturbations used in image processing tasks, and these exclusively consists of modifying pixels and groups of pixels to determine their usefulness to the detection.
In the following, I will discuss two variations of this standard perturbation-based approach.

\paragraph{Game theoretic approach with perturbation}\label{par:game-theoretic-approach-with-pertubation}

 is a model-agnostic interpretation method that employs cooperative game theory to explain model predictions.
This translates to decomposition of the input image into a number identical, equally-sized components, referred to as \("\)features\("\).
These are processed further and a so called Shapley value is calculated to each feature, based on their contribution to the prediction.
This will be projected onto the image, where each feature will be coloured according to this value, thus facilitating easier reading and comprehension.
A detailed examination of this process can be found in reference \ref{par:shap}, where the operational mechanics of this methodology will be elucidated.

Based on the work of Lundberg~\cite{lundberg2017unifiedapproachinterpretingmodel} and Liang~\cite{LIANG2021168},
this method, namely SHAP (SHapley Additive exPlanations), is particularly useful for image processing tasks, as it provides detailed insights into the model's decision-making process.
On grounds of the aforementioned, I chose to use SHAP in my project, as it offers a comprehensive and theoretically grounded approach to interpreting the model's predictions.

\paragraph{Adversarial-based interpretation}\label{par:adversarial-based-interpretation}

methods, based on the work of Lian~\cite{LIANG2021168},
represent a subtype of perturbation methods,
the objective of which is to provide more robust interpretations.
These methods address a common issue in deep neural networks (DNNs), namely poor generalisation performance, as highlighted in the base of my work~\cite{LIANG2021168}.
Deep neural networks (DNNs) are highly sensitive to perturbations in input data, which can result in significant alterations to the predictions made,
thus affecting the stability and reliability of the interpretations produced.
The earliest adversarial-based interpretation methods, as exemplified by the approaches put forth by Fong et al. (2017) and other researchers,
employed techniques to mitigate the impact of perturbations.
This was achieved by utilising random masks and regularising the masks with total-variation norms to smoothen the images that had been perturbed.

Although these methods enhance robustness, they necessitate manual tuning of hyperparameters and yield interpretations of relatively low resolution,
thereby constraining their capacity for fine-grained analysis.
On these grounds, they are not suitable for applications that require high-resolution interpretations, and such I did not use them in my project.

\subsubsection{Concept-based Interpretation}\label{subsubsec:concept-based-interpretation}

Concept-based interpretation methods, based on the work of Liang~\cite{LIANG2021168}, employ predefined sets of human-understandable concepts to provide explanations for deep learning models.
The generation of these concepts is typically conducted with the assistance of domain experts, thereby rendering this method particularly
useful for interpreting the decision-making processes of models in a manner that aligns with human intuition.
The process commences with the selection of a concept set (\(C\)), which encompasses images or data that correspond to specific attributes of the input.
For instance, the concept set may include images of tiger stripes for an image of a tiger.

The definition of these concepts is facilitated by humans, and the concept set is subsequently incorporated into the deep neural
network (DNN) in order to ascertain which concepts are most relevant to the model's predictions.
Two prominent concept-based interpretation methods are Testing with Concept Activation Vectors (TCAV) and Network Dissection (ND).
TCAV explains the behaviour of the model in question by employing human-defined concepts, namely Concept Activation Vectors (CAVs)
, which quantify the importance of said concepts to the model's predictions through directional derivatives.
This approach allows for the assessment of the sensitivity of the model's predictions to specific concepts,
such as texture or colour, across a range of inputs.

In contrast, Network Dissection quantifies the relationship between internal neural network features and visual concepts,
utilising metrics such as Intersection over Union (IoU) to ascertain the degree to which individual neurons correspond to
concepts such as colour, texture, or objects. This approach facilitates a comprehensive understanding of the specific layers
and neurons in a CNN that are responsible for detecting various concepts, thereby offering insights into the interpretability of each layer.

Based on the work of Liang~\cite{LIANG2021168}, these methods are particularly useful for image processing tasks,
as they provide human-understandable explanations for the model's predictions.
However, they require the input of domain experts to define the concepts, which can be time-consuming and may introduce bias.
Furthermore, the interpretability of these methods is limited by the quality of the concept set, which may not capture all
the relevant features of the input data.

As a result, I found that these methods were not suitable for my project, as they
require a high level of domain expertise and may not provide the level of interpretability required for fine-grained analysis.





\subsubsection{Comparison of chosen Model-Agnostic Methods}\label{subsubsec:comparison-of-model-agnostic-methods}

In this subsection, I will compare two of the most popular model-agnostic interpretation methods, LIME and SHAP.
Based on the work of Liang~\cite{LIANG2021168}, these model-agnostic interpretation methods have their strengths and weaknesses.
Through their example I will illustrate the differences between the two methods.

LIME and SHAP while using
different approaches, there are both using some form of perturbation to explain the model's predictions.
LIME approximates the model locally by fitting a simple interpretable model around
the prediction of interest, providing local explanations by perturbing the input
data and observing the changes in the model's predictions.
It is generally faster and simpler, making it suitable for quick, local explanations.


On the other hand, SHAP is based on cooperative game theory and uses Shapley values to attribute the contribution of
each feature to the prediction.
It provides both local and global explanations by considering all possible combinations of features,
producing consistent and theoretically sound explanations.


Based on the discussions by~\cite{lundberg2017unifiedapproachinterpretingmodel}, Kernel SHAP, a variant of SHAP,
is particularly useful for image processing tasks, as it can handle high-dimensional data efficiently.
It is based on the idea of approximating the model with LIME and using the Shapley values to explain
the predictions the model made.
This approach combines the strengths of both LIME and SHAP, providing accurate and efficient explanations for
image processing models.

However, SHAP is more computationally intensive due to the necessity of considering all potential feature combinations. During the execution of the interpretation scripts, SHAP exhibited approximately four times the GPU memory usage of LIME, while the runtime was approximately one hundred times longer. While LIME is flexible and can be applied to any model and data type, SHAP offers a more comprehensive and theoretically grounded approach at a higher computational cost.

\subsection{Model-Specific Methods}\label{subsec:model-specific-methods}
The base idea behind this category is to ground our interpretation on the internal state of the model.
By this in the field of image processing, it aims to somehow visualize some parameters of the inner state of the model, and project it back to the input image.
Multiple methods exist in this category, such as Class Activation Maps, Gradient-based methods.
The following paragraph, present a discussion of the Class Activation Maps with reference to the work of Bany Muhammad and his introduction of his interpretation method~\cite{Muhammad_2020}.
Its implementation will be discussed in the next section.
Furthermore, I will discuss the Gradient-based methods based on the work of Selvaraju titled \("\)Grad-CAM: Visual
Explanations from Deep Networks via Gradient-Based Localization\("\)~\cite{Selvaraju_2019}.




\subsubsection{Class Activation Maps}\label{subsubsec:CAM}

Class Activation Maps (CAMs) are a model-specific interpretation method designed to highlight the regions or features
in an image that are most pertinent to a given model.
In an image, these are the regions that are most relevant to a model's decision, particularly in the context of classification tasks.
The operation of CAMs is based on the projection of the activations from the convolutional layers back onto the input image.
This results in the generation of a heatmap, visually represents the areas that contributed
the most to the prediction~\cite{Muhammad_2020}.
This method provides valuable insights into which features the model is focusing on, enabling a better understanding of
the model's decision-making process.

The original CAM technique, first introduced by Zhou et al. (2016),
is specifically designed for models that include global average pooling layers before the final output layer,
often referred to as the \("\)head\("\).
These pooling layers facilitate and are responsible for the smooth projections (feature maps) on the methods output image.
While the CAM technique is highly effective, it is somewhat limited in that it is optimized for this specific architectural
configuration, which restricts its versatility.
It is less flexible for networks that rely on fully connected layers following the convolutional layers~\cite{Zhou_2016}.

To illustrate, in a classification task where a model identifies a vehicle in an image, CAM would generate a heat map over and around the vehicle.
This indicates that the model primarily focused on that object when making its decision, which is a useful insight.
In the absence of this information, there is a possibility that the network may be detecting a feature that is not
exclusive to that particular object class and this can result in greater confusion within the model.

This type of visual feedback is especially valuable in domains such safety critical (automotive, etc.) or medical fields,
where understanding why the model identifies certain patterns is critical~\cite{Muhammad_2020}.

In response to the architectural limitations of the original CAM method, several improvements have been proposed.
One significant extension is Grad-CAM (Gradient-weighted Class Activation Mapping), which eliminates the dependence
on specific model architectures by using gradients to compute the heatmaps.
Details of Grad-CAM will be further discussed in the next section~\cite{Selvaraju_2019}.

\subsubsection{Gradient-based Methods}\label{subsubsec:gradient-based-methods}

Gradient-based methods, such as Grad-CAM (Gradient-weighted Class Activation Mapping),
take a different approach by utilizing the gradients flowing through the backpropagation of
the network. It can be used to localize the features within the image, that the models learn on.
Grad-CAM generates heatmaps that highlight the regions of the image that contributed the most to the model's output.
Selvaraju et al. (2019) demonstrated the effectiveness of this technique in their work titled
\("\)Grad-CAM: Visual Explanations from Deep Networks via Gradient-Based Localization\("\)~\cite{Selvaraju_2019}.

By leveraging gradients, Grad-CAM provides more precise localization of features compared to basic CAM techniques,
making it particularly useful for deep neural networks in image recognition tasks.

At the end, this method, despite of being more accurate than the vanilla EigenCAM, I did not use it in my project due
to two main considerations.

First of all, the method is more computationally expensive, and requires more memory to run.

Secondly, the method is more complex to implement and requires more input from the model's internal state.
In order to ensure the model calculates the gradients correctly, it should be fed with the whole back-propagation chain,
which has made the other interpretation methods more time-consuming to implement and run.

Based on that I chose to use the CAM method in my project, as it is simpler to implement and provides valuable insights into the model's decision-making process.

\newpage
\subsubsection{Comparison of Model-Specific Methods}\label{subsubsec:comparison-of-model-specific-methods}

Both CAM and Grad-CAM methods have been widely adopted for interpreting convolutional neural networks
and are valuable tools for visualizing and understanding model behaviour.
While CAM is limited by its dependence on specific model architectures, Grad-CAM offers a more flexible and
generalizable approach that can be applied to a wide range of models.
The use of gradients in Grad-CAM allows for more precise localization of features within the image,
providing valuable insights into the decision-making process of the model.

In contest of the resulting visualizations, the output of the two networks tends to be really similar to each other.
Which can be because the two methods are based on the same principle, and the implementation of the Grad-CAM
is a direct extension of the CAM method, utilizing the activation maps with the gradients to provide the explanation.

\subsection{Comparison of Interpretation Methods}\label{subsec:evaluation-interpretation-methods}

In this section, I will compare the model-agnostic and model-specific interpretation methods discussed and chosen in the previous subsections
and paragraph.
Model-agnostic and model-specific interpretation methods each offer unique strengths and weaknesses when applied to
an image processing model.
Both approaches aim to provide insights into the model's decision-making.
However, they are fundamentally different in terms of their underlying assumptions, flexibility, and computational requirements.


Model-agnostic methods often operate by perturbing the input data, observing the changes in the  predictions of the model,
and deriving explanations based on these observations as discussed previously~\ref{subsubsec:pertubation-based-interpretation}.
This makes them highly flexible and applicable to a wide variety of tasks and models.
However, this flexibility sometimes comes at the cost of interpretability precision, as model-agnostic methods approximate
how a model behaves based on perturbations, which may not capture the complete depth of the model’s decision logic.

Conversely, model-specific methods are designed to align with the internal mechanisms of a specific model architecture,
thereby facilitating the generation of explanations based on the model's inherent operational logic.
These methods are often more precise, as they can directly access the model's components — such as layers,
weights, activation vectors, or gradients and utilize this information to interpret how the model made its decision.
In fields like image processing, model-specific methods, such as Class Activation Maps (CAMs) or Gradient-based approaches,
provide visual insights into the areas of an image that most influenced the model’s predictions.
By visualizing the internal state of the model, model-specific methods offer a more direct way of understanding the behaviour of the model,
especially in tasks where identifying critical features or regions of the data is essential.

In comparing these two categories, the fundamental difference lies in their generalization versus precision.
Model-agnostic methods are particularly useful because of their ability to work across different models and domains,
making them versatile tools in scenarios where multiple model types and architectures are used.
However, they frequently offer a higher-level view of the behaviour of the model, which may be less precise than the
insights gained from model-specific methods.
Conversely, model-specific methods provide a more detailed data for understanding of the decision-making process,
directly interacting with the internal components.
This allows for more detailed insights but limits the applicability to certain model types.







\section{Model interpretation using external solutions}\label{sec:model-interpretation-application}
%! Author = NyPeter
%! Date = 2024. 10. 13.



%\section{Model interpretation using external solutions}\label{sec:model-interpretation-%application}
\section{Model interpretation component}\label{sec:model-interpretation2}
Implementing the second component, which was responsible for the explainability of the model, my tasks were to implement and use the interpretation methods discussed in the next chapter.
In order to run them I had to choose data, to feed to the models input.
I chose the test set's pictures taken in the german city of Bonn,
because it was the smallest set of pictures in the test set, which set was not run on the model before, to optimise the runtime.

The implementation of the different methods were discussed in their own paragraphs in the next subsection.

After the successful implementations of these methods, 2 of them got implemented in the main XAI python script. Because of hardware limitations I decided to use an external, cloud solution to run SHAP.

After that I ran the python script to produce results for LIME and EigenCAM. Then on the preprocessed pictures and model I ran the notebook containing SHAP, and its output was downloaded into my local machine.

\subsection{Methods for model interpretation}\label{subsec:methods-for-model-interpretation}

This component was implement into a python script file, which loaded the the model from a \("\).pt\("\) PyThorch\footnote{https://pytorch.org} file and reads in the given image.
After that process is complete, a quick resizing of the images is taking place, then the different interpretation methods are run after each other.

\paragraph{EigenCAM:}\label{par:eigencam}
this model-specific interpretation method employs the eigenvalues of the convolutional layers to generate class activation maps.
The eigenCAM values are derived from the activation vectors of the various layers, which are employed in the calculation of the activation maps.
Subsequently, the maps are projected back to the input image, thereby highlighting the regions that are of particular importance for the model's prediction.

The aforementioned regions are useful for understanding the output of each layer and the decision-making process of the model.

The output of the EigenCAM method can be interpreted as a heatmap, where the intensity of the colour represents the importance of
the region for the model's prediction.
By analysing these heatmaps for different layers of the same image, we can gain a deeper insight into the model's decision-making process.

I used the github repository of Jacob Gildenblat ~\cite{jacobgilpytorchcam}
, which contains an implementation of the method proposed by Zhou\cite{Zhou_2016}.
After that the needed functions got called in the main python script, on the preprocessed image and the loaded model, and got iterated over multiple layers.

\paragraph{Local Interpretable Model-agnostic Explanations:}\label{par:lime}

this model-agnostic, perturbation-based method employs a local surrogate model to approximate the behaviour of the original model, by
approximating the model's decision boundary by perturbing an instance and fitting a simpler interpretable model (such as linear regression) to the modified samples
The method is frequently employed for the purpose of elucidating individual predictions and elucidating the model's decision-making process. For instance, in image classification, LIME can identify which pixels or regions(segments or superpixels) contribute most to the model's decision, while in other uses, for example in texts, it can highlight important words or phrases.

In my implementation of XAI component,  superpixels are generated for LIME, using the Slic algorithm, and a linear model is fitted to the perturbed data.
The superpixel perturbation represents a more sophisticated approach than regular grid-based perturbations, as it is more likely to capture the relevant features of the image.
This is achieved by creating a non-regular shaped superpixel out of similar and/or approximate pixels.
Subsequently, the linear model, which is a white-box mode, is employed to approximate the behaviour of the original model and provide an explanation for the prediction.

The output of the LIME method is a set of coefficients that represent the importance of each feature for the prediction of the model.
By employing these coefficients and superpixels, the superpixels that are most crucial for the prediction can be identified by the
colour of the given superpixels on the methods output image.
Useful superpixels are green, while harmful superpixels, which impede the detection of objects, are red.

To implement and integrate this method, I wrote a small script, which used the LIME python package\footnote{https://pypi.org/project/lime/} to help with the correct implementation of LIME, after that I called the function containing all the features regarding LIME explanation in the main interpretation component called Yolov8\_XAI.py.

\paragraph{Shapley Additive explanations:}\label{par:shap}

this model-independent, perturbation-based method employs cooperative game theory to assign an \("\)importance\("\) value,
known as the Shapley value, to each feature, representing its contribution to the model's prediction.
This approach is more intricate but nevertheless satisfactory than the LIME method.
The image is partitioned into equal-sized segments, which collectively form a grid, these segments are then organised into a set.
This set and all its subsets gets perturbed individually and the model's prediction is observed on them, based on these predictions
they get assigned a Shapley value.
This value is getting calculated from iteration to iterations for each segment, until all the subsets are evaluated.
This calculation is based on the Shapley value formula.

In the implementation, I used Kernel SHAP, which can be described as Linear LIME combined with Shapley values calculated by the Kernel SHAP algorithm.

\begin{theorem}[Shapley kernel]
\end{theorem}
\label{corr:shap_kernel}
The equations below are from \cite{lundberg2017unifiedapproachinterpretingmodel}.
\begin{equation}
\begin{split}
\Omega(g) &= 0 \text{: baseline value for model g,}\\
\pi_{x'}(z') &= \frac{(M-1)}{(\binom{M}{|z'|}) |z'| (M - |z'|)}\text{: weight assigned to each subset of z',} \\
L(f,g,\pi_{x'}) &= \sum_{z' \in Z} \left[ f(h_x^{-1}(z')) - g(z') \right]^2 \pi_{x'}(z') \\
\label{eq:kernel}
\end{split}
\end{equation}
\vspace{-0.4cm} \\where $|z'|$ is the number of non-zero elements in $z'$.

The loss function is employed to quantify the discrepancy between the model's prediction and that of the surrogate model.

The loss is calculated by taking the squared difference between the predictions of (f) and (g) for each subset (z').



As this method is model-agnostic, it can be employed to interpret the predictions of any model, regardless of its complexity.
his can be a significant limitation when interpreting complex models with a large number of features.

Thus, the method was only operational on an online platform, Google
Colab\footnote{\\https://tinyurl.com/shapColab}, through a Jupiter notebook environment, as it provides
the necessary computational resources and a fitting platform for the SHAP algorithm to run efficiently.
The part of the implementation is based on the notebook of Akshay Gupta\footnote{https://github.com/akshay-gupta123/Face-Mask-Detection/blob/main/Notebooks/Shap.ipynb}, his code was rewritten and repurposed, mainly the structure of the code remained, the most heavy changes were made in the image processing, the output generation and the parameterisation of the used kernel SHAP function.

%Methods for interpretation: LIME, SHAP
\subsection{Evaluation of interpretation methods}\label{subsec:evaluation-of-interpretation-methods}
%Results on small_vehicle
% Képmagyarázat


In all of the paragraph, the chosen methods will try to explain on the image on the Figure\ref{fig:bonn35}.
\begin{figure}[h]
    \centering
    \includegraphics[width=1\linewidth]{figures/bonn_000035_000019_leftImg8bit_original}
    \caption{Picture Bonn35 }
    \label{fig:bonn35}
\end{figure}

On which the models prediction can be observed in the image: Figure\ref{fig:detbonn35}.
\begin{figure}[h]
    \centering
    \includegraphics[width=1\linewidth]{figures/bonn_000035_000019_leftImg8bit}
    \caption{Model's detection to the picture Bonn35 }
    \label{fig:detbonn35}
\end{figure}

In the followings I will describe all the image interpretations given by the different interpretation methods. And try and interpret the model's workings based on those outputs.

\newpage

\subsubsection{Evaluation of the EigenCAM method}\label{subsubsec:evaluation-of-the-eigencam-method}

In this figure constellation of Figure\ref{fig:Bonn_000035_000019} different layers are presented.
These layers are

\begin{table}[h]
    \centering
    \begin{tabular}{|c|c|p{10cm}|}
        \hline
        \textbf{Name} & \textbf{Place} & \textbf{Description} \\
         \hline
         -2 C2f  & Feature Pyramid & A composite layer with Cross-Stage Partial Network (CSP) structure, designed to increase gradient flow and reduce memory usage by splitting the feature maps and merging them at the end. \\
         \hline
         -3 Concat & Feature Aggregation & Concatenates feature maps from different scales or layers to merge information, enabling multi-scale predictions. \\
         \hline
         -4 Conv & Convolutional Layer & A standard 2D convolution layer that extracts features from the input by applying filters to generate feature maps. \\
         \hline
         -5 C2f  & Backbone Stage & Similar to the C2f layer at -2, it is responsible for feature extraction with a CSP-like structure to improve computational efficiency and gradient flow. \\
         \hline
    \end{tabular}
    \caption{Yolov8 architecture layers and their descriptions.}
    \label{tab:yolov8_layers}
\end{table}

\begin{figure}[h]
    \centering


    \begin{subfigure}[b]{0.47\textwidth}
        \centering
        \includegraphics[width=\textwidth]{figures/bonn_000035_000019_leftImg8bit.pnglayer-2/bonn_000035_000019_leftImg8bit.png_object(0)_heatmap}
        \caption{Layer -2}
        \label{fig:a-2}
    \end{subfigure}
    \hfill
    \begin{subfigure}[b]{0.47\textwidth}
        \centering
        \includegraphics[width=\textwidth]{figures/bonn_000035_000019_leftImg8bit.pnglayer-3/bonn_000035_000019_leftImg8bit.png_object(0)_heatmap}
        \caption{Layer -3}
        \label{fig:-3}
    \end{subfigure}\\
    \hfill
    \begin{subfigure}[b]{0.49\textwidth}
        \centering
        \includegraphics[width=\textwidth]{figures/bonn_000035_000019_leftImg8bit.pnglayer-4/bonn_000035_000019_leftImg8bit.png_object(0)_heatmap}
        \caption{Layer -4}
        \label{fig:-4}
    \end{subfigure}
    \hfill
    \begin{subfigure}[b]{0.49\textwidth}
        \centering
        \includegraphics[width=\textwidth]{figures/bonn_000035_000019_leftImg8bit.pnglayer-5/bonn_000035_000019_leftImg8bit.png_object(0)_heatmap}
        \caption{Layer -5}
        \label{fig:-5}
    \end{subfigure}
    \hfill

    \caption{Activation maps for the Layer -2, -3, -4, -5 of Bonn35}
    \label{fig:Bonn_000035_000019}
\end{figure}

The images can be read and interpreted by identifying the layer responsible for each process and observing the colours on the provided activation maps.
The colour red represents the lowest value of activation, while blue represents the highest.
The range of layers observed spanned from the second to last layer to the fifth to last layer,
allowing for the finest resolution of the head to be observed.
This enabled the creation of a visualisation of the areas that were particularly useful in the detection process.

It is evident that the highest activation values were present in the regions where traffic-related objects were identified.
Pedestrians were observed to have their outlines and legs presented with higher activation values.
Additionally, it was noted that the rough edges from the more inner layers in the network exhibited a process of smoothening layer to layer.

\subsubsection{Evaluation of the LIME method}\label{subsubsec:evaluation-of-the-lime-method}
In the figure constellation of Figure\ref{fig:LIME1} and~\ref{fig:LIME2} different types of LIME functions are presented.
These are LIME\_SUM and LIME\_MULTI\@.
LIME\_MULTI aggregates the contributions of each feature across multiple instances by summing the importance scores.
This approach provides a simplified, overall view of how certain features influence the decisions of the model on average,
making it easier to identify the most significant features contributing to the predictions in a broad sense.
LIME\_SUM, on the other hand, focuses on explaining predictions at a finer level.
It applies LIME to multiple instances individually and then analyses the distribution of feature importance across these instances.
This method captures more granular information, allowing for a more detailed understanding of feature interactions and their varying contributions across different samples.

\begin{figure}[h]
    \centering
    \includegraphics[width=\textwidth]{figures/best-box_bonn_000035_000019_leftImg8bit}
    \caption{LIME\_SUM}
    \label{fig:LIME1}
\end{figure}
\hfill
\begin{figure}[h]
    \centering
    \includegraphics[width=\textwidth]{figures/best-box_bonn_000035_000019_leftImg8bit_MULTI}
    \caption{LIME\_MULTI}
    \label{fig:LIME2}
\end{figure}

The method in question lacks the required granularity and is unable to keep up with more complex neural networks.
Consequently, the resulting output is of questionable reliability, exhibiting inconsistencies such as the appearance of greens adjacent to detected objects, multiple artifacts, and a lack of discernible pattern or rationale in the red markings.
This may be attributed to the method's inherent limitations in providing accurate explanations for intricate problems.

\subsubsection{Evaluation of the SHAP method}\label{subsubsec:evaluation-of-the-shap-method}
In the figure\ref{fig:SHAP_result} SHAP is run to the image, and the shap values are clearly visible on the pictures, each picture represents the run on one instance of detected object.

\begin{figure}[h]
    \centering
    \begin{subfigure}[b]{\textwidth}
        \includegraphics[width=\textwidth]{figures/output1}
        \caption{Shapley values of foreground objects}\label{fig:SHAP_results11}
    \end{subfigure}
    \hfill
    \begin{subfigure}[b]{\textwidth}
        \includegraphics[width=\textwidth]{figures/output1,2}
        \caption{Shapley values of background objects}\label{fig:SHAP_results12}
    \end{subfigure}
    \hfill
    \caption{SHAP results for different instances in Bonn35}
    \label{fig:SHAP_result}
\end{figure}


Red representing higher positive SHAP values, while blue represents negative SHAP values.
Considering the most of the pictures, especially where overrepresented, highly visible objects can be seen, the interpretation is clean and readable.


The SHAP results in both rows (on Figure\ref{fig:SHAP_result} a and b) provide the following insights into the behaviour of the model in this traffic scenario:

\begin{itemize}
\item The uniform highlighting of pedestrians and vehicles across all images indicates that the model is effectively identifying and focusing on the critical elements within the traffic scene. This is an encouraging indication of the interpretability of the model and reliability in detecting important objects.
\item The absence of emphasis on or attention to the muted or cooler blocks on buildings and other background areas indicates that the model does not rely on these features for its predictions. This is an anticipated and favourable outcome in the context of traffic-related tasks.
\item Additionally, an anomaly is is evident on the more \("\)noisy\("\) images, which may be a limitation of this methodology.
\end{itemize}




\subsection{Addressing the Anomaly regarding the noise of SHAP and the probability of the models detection}\label{subsec:Addressing the Anomaly regarding SHAP and the probability of detection}
% Detekció bizonyossága vs interpretáció zajosság

As previously stated in the previous subsection~\ref{subsec:evaluation-of-interpretation-methods},
the SHAP values for objects with lower detection probability are characterised by a high degree of noise,
rendering them unable to provide a clear and coherent explanation for the predictions of the model.
This anomaly can be attributed to the low probability of detection for this instance pedestrian class, which results in unreliable,
often contradicting information for the SHAP algorithm to interpret.
This phenomenon is common in object detection models, where the detection accuracy for small objects is lower than that for larger or more clearly visible objects.
Furthermore, as previously discussed in section~\ref{par:adversarial-based-interpretation} that the detection threshold of this model and the probability of detection can
have a significant impact on the actual decision-making process of the model.
The combination of perturbation-based interpretation methods may result in unreliable and noisy results, as the perturbed image segments may be misclassified by the model.

Furthermore, an additional cause of this noise can be identified, which is more comprehensible when considering the proportions of the various classes of traffic objects present in the dataset.
As illustrated in Figure~\ref{fig:Label_distribution} the dataset exhibits a significant class imbalance, with certain classes, such as vehicles, being overrepresented in comparison to others, such as pedestrians.
This imbalance may result in the generation of noise in the interpretations, as the model may exhibit a tendency to favour the more frequent classes due to the larger volume of data from which it has learned.
Consequently, the limited number of examples for less frequent classes, such as pedestrians, may not provide the model with sufficient information to effectively capture their distinctive and often subtle features.

\begin{figure}[ht]
    \centering
    \includegraphics[width=0.8\linewidth]{figures/labels-30}
    \caption{Label distribution}
    \label{fig:Label_distribution}
\end{figure}

This deficiency in the training data can give rise to a number of issues.
For example, the model may encounter difficulties in generalising effectively when it encounters pedestrian objects in real-world scenarios, due to insufficient exposure to their diverse appearances and contexts during training.
Consequently, the model may miss-classify or fail to identify pedestrians, which can introduce additional noise in its output, manifesting as false negatives or incorrect predictions.(This is represented on Figure~\ref{fig:Confusionmatrix})
Furthermore, the distinctive characteristics that differentiate pedestrian objects, such as specific shapes, movements, or interactions with the environment, may be under-represented in the model's learned representations.
To mitigate this noise and enhance detection accuracy, it is essential to balance the dataset or augment it with more diverse examples of the minority class, thereby enhancing the model's ability to learn these crucial features effectively.

\begin{figure}
    \centering
    \includegraphics[width=1\linewidth]{figures/confusion_matrix2}
    \caption{Confusion matrix showing the us the detailed classification results of an algorithm on a test set by analysing its rows, columns, or entries}
    \label{fig:Confusionmatrix}
\end{figure}
%% külön cm-ről

The provided image presents two confusion matrices, one with absolute values and the other with percentages, which elucidate pivotal aspects of the model's performance.

To address this issue, an increase in the detection threshold of the model for the class of noisy object  may be beneficial in improving the detection probability of already detected objects.
This could be achieved by ignoring detections with a lower probability, thus enhancing the reliability and decreasing the noise
of SHAP interpretations.
This adjustment can help reduce the noise but may also result in the exclusion of some objects from the interpretation process.
Making the interpretation process more incompletely, but more reliable and less noisy.

\subsubsection{Explanations to the Confusion Matrix}
The confusion matrices provide a detailed assessment of the model's classification performance for categories: \textit{small\_vehicle}, \textit{person} and \textit{large\_vehicle}.

The absolute score matrix (left) shows high accuracy for \textit{small\_vehicle} (3,780 correct) and \textit{person} (2,820 correct), but also reveals misclassifications, such as 640 \textit{small\_vehicle} instances identified as \textit{background}. Similarly, the class \textit{two-wheeler} has 928 correct classifications, but some confusion with \textit{background}.

The percentage matrix (right) normalises these results and shows high accuracy for \textit{small\_vehicle} (85.06\%) and \textit{person} (77.49\%), while \textit{two wheeler} accuracy drops to 68.74\% due to misclassification (30.29\% as \textit{background}). The \textit{background} category also shows considerable confusion with other classes.

These matrices highlight strengths and weaknesses, guiding improvements in preprocessing, model design, and training while enhancing result interpretability—a core focus of this thesis.
\newpage





% Summary
%~~~~~~~~~~~~~~~~~~~~~~~~~~~~~~~~~~~~~~~~~~~~~~~~~~~~~~~~~~~~~~~~~~~~~~~~~~~~~~~~~~~~~~
\section*{Summary}
\addcontentsline{toc}{chapter}{Summary}

This document presents a comprehensive study on the training and interpretation of a neural network
model for traffic object detection.
The key points discussed in the chapters are summarized as follows:

\begin{itemize}
    \item **Model, Training, and Data**: The YOLOv8 model architecture, including its backbone, neck, and head components, is detailed. The training process and the Cityscapes dataset used for training are also discussed.
    \item **Model Interpretation Using External Solutions**: The importance of model interpretation is highlighted. Various model-agnostic methods such as LIME and SHAP, as well as model-specific methods like EigenCAM and EigenGradCAM, are explained.
\end{itemize}

The study concludes that effective training and interpretation of neural network models are crucial for accurate and reliable traffic object detection.





% Acknowledgements
%~~~~~~~~~~~~~~~~~~~~~~~~~~~~~~~~~~~~~~~~~~~~~~~~~~~~~~~~~~~~~~~~~~~~~~~~~~~~~~~~~~~~~~
%%----------------------------------------------------------------------------
\chapter*{\koszonetnyilvanitas}\addcontentsline{toc}{chapter}{\koszonetnyilvanitas}
%----------------------------------------------------------------------------

I would like to express my deepest gratitude to my parents, Dr. Krisztián Nyilas and Ildikó Nyilasné Mészáros, for their unwavering support and encouragement throughout this journey. I am also sincerely thankful to my girlfriend, Anett Bakos, whose understanding and help with  helped me stay focused.

Special thanks to my consultant, Dr. Gábor Hullám, for his invaluable guidance and insights, which greatly enriched this work.

I am profoundly grateful to all who have supported me along the way.


% List of Figures, Tables
%~~~~~~~~~~~~~~~~~~~~~~~~~~~~~~~~~~~~~~~~~~~~~~~~~~~~~~~~~~~~~~~~~~~~~~~~~~~~~~~~~~~~~~
%\listoffigures\addcontentsline{toc}{chapter}{\listfigurename}
%\listoftables\addcontentsline{toc}{chapter}{\listtablename}


% Bibliography
%~~~~~~~~~~~~~~~~~~~~~~~~~~~~~~~~~~~~~~~~~~~~~~~~~~~~~~~~~~~~~~~~~~~~~~~~~~~~~~~~~~~~~~
\addcontentsline{toc}{chapter}{\bibname}
\bibliography{bib/mybib}


% Appendix
%~~~~~~~~~~~~~~~~~~~~~~~~~~~~~~~~~~~~~~~~~~~~~~~~~~~~~~~~~~~~~~~~~~~~~~~~~~~~~~~~~~~~~~
%include{content/appendices}

%\label{page:last}
\end{document}
