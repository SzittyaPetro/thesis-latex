\pagenumbering{roman}
\setcounter{page}{1}

\selecthungarian

%----------------------------------------------------------------------------
% Abstract in Hungarian
%----------------------------------------------------------------------------
\section*{Kivonat}\addcontentsline{toc}{chapter}{Kivonat}

Ez a diplomamunka egy mesterséges intelligencia alapú rendszer fejlesztését mutatja be, amely képes magától felismerni és osztályozni különböző dinamikus közlekedési objektumokat képeken, illetve a dolgozat foglalkozik még a neurális hálózat interpretálásával.

A projekt a Yolov8 modelt használja, amely egy konvolúciós mély neurális hálózat alapú objektumfelismerő algoritmus.
Modell interpretálására modellfüggő és modellfüggetlen megoldások elméleti hátterével foglalkozik, majd választ három megoldást, amelyek a LIME, SHAP és az EigenCAM lettek.

A dolgozat részletesen tárgyalja a modell adatkezelésének,betanításának és interpretálásának folyamatát, valamint értekezik a gyakorlatot alátámasztó elméletről és elért eredményekről. Beszél a különböző magyarázómódszerek hibáiról
majd levonja a következtetéseket és új kutatási irányokat javasol a következő munkának.

\vfill
\clearpage
\selectenglish
%----------------------------------------------------------------------------
% Abstract in English
%----------------------------------------------------------------------------
\section*{Abstract}\addcontentsline{toc}{chapter}{Abstract}

This thesis paper presents the development of an artificial intelligence-based system that can self-detect and classify different dynamic traffic objects in images, and the paper also deals with the interpretation of the neural network.

The project uses the Yolov8 model, which is a convolutional deep neural network based object recognition algorithm.
For model interpretation, model dependent and model independent solutions are discussed with theoretical background and then three solutions are chosen which are LIME, SHAP and EigenCAM.

The paper discusses in detail the process of model data processing, training and interpretation, and discusses the theory and results that underpin the practice. It discusses the flaws of the different interpretation methods
then draws conclusions and proposes new research directions for future work.

\vfill
\cleardoublepage

\selectthesislanguage

\newcounter{romanPage}
\setcounter{romanPage}{\value{page}}
\stepcounter{romanPage}