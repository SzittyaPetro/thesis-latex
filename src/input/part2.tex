%! Author = NyPeter
%! Date = 2024. 10. 13.

\section{Model Interpretation}\label{sec:model-interpretation}

\subsection{Importance of Interpretation}\label{subsec:importance-of-interpretation}

Interpreting machine learning models is essential for understanding their decision-making processes.
It helps in identifying biases, improving model performance by providing information helping us the augment data more efficiently, and building trust with users.
Interpretation techniques provide insights into how image processing models make predictions and highlight the most influential features.

Interpretation of machine learning algorithms is a fairly new field, but it has gained significant attention
in recent years due to the increasing complexity of models and the need for transparency and accountability.
This need comes form industrial and judicial actors, who require explanations for the decisions made by models,
especially in safety-critical applications such as autonomous vehicles, healthcare, and finance.
In the last couple of years, the field of model interpretation has seen significant advancements, each brought
us closer to understanding the inner workings of machine learning models, and try to put reason and logic behind
otherwise black-box models.

Furthermore, based on these advancements the aforementioned actors are more willing to adopt laws,
standards and regulations that require models to be interpretable,
if they are to be used in non-research areas.

Building on the aforementioned state of the machine learning world, I felt it was important to try
and create a fairly coherent interpretation of the model to meet today's industry standards.

\subsection{Introduction to Interpretation methods}\label{subsec:introduction-to-interpretation-methods}

Model interpretation is a critical aspect of machine learning that aims to explain the decision-making process of models. % Ismétlés
Interpretable models are essential for building trust with users, identifying biases, and with this information,
developers have a deeper understanding that can help them improve model performance.
In this section, there will be a discussion of the importance of model interpretation and a review of different methods of interpretation.

In their survey~\cite{LIANG2021168}, Liang et al. highlight the significance of model interpretation techniques in providing insights from the model itself, and the output of the model, without knowing the inner state of the model.

Therefore they sorted these techniques into two categories:
\begin{enumerate}
    \item data driven or more commonly as model-agnostic methods
    \item model driven or model-specific methods
\end{enumerate}

The following sections will provide an overview of the two categories.
In the event that one is used, an analysis of the algorithms will be presented in light of the implementation used in this project.
This will include a comparison and contrast of their respective strengths and weaknesses.



\subsection{Model-Agnostic Methods}\label{subsec:model-agnostic-methods}
Based on the work of  Liang~\cite{LIANG2021168}, model-agnostic interpretation methods can be divided into three categories:
\begin{enumerate}
    \item Perturbation-based interpretation and a Game theoretic approach
    \item Adversarial-based interpretation
    \item Concept-based interpretation
\end{enumerate}


As discussed in the work of Liang~\cite{LIANG2021168}, these are designed to explain model predictions without relying
on the internal structure of the model, hence their name, making them applicable to a wide range of models.
This insight may be more applicable to local examples, as the approach only provides explanations for the given data and not globally for the model itself.
For image processing purposes, these methods are particularly easy to understand and implement.

Both the game-theoretic and perturbation-based interpretation methods involve altering the input data
and observing the resulting changes in the model's predictions on the modified image, while Adversarial-based interpretation methods focus on generating small perturbations or adversarial examples that can significantly change the model's predictions, highlighting the model's vulnerabilities.

On the other hand, concept-based interpretation methods aim to link model behavior to high-level human-understandable concepts.

Regardless, I will discuss the different types of data driven interpretation methods in the following paragraphs, each on the basis of Liang's work~\cite{LIANG2021168}

\subsubsection{Perturbation-based Interpretation}\label{subsubsec:pertubation-based-interpretation}

Perturbation-based interpretation methods are model-agnostic techniques that explain model predictions by perturbing the input data.
These methods generate perturbed samples by introducing noise (or masking portions of the image) to the input image and observing the alterations in the model's predictions.
This masking is the main principle behind these methods.

The most common type of masking is occlusion, where parts of the image are covered to determine their importance for the model's output.
It can be interpreted as a form of feature selection, where the model's output is evaluated based on the presence or absence of specific features.

In essence, perturbation-based methods can be conceptualised as an optimisation problem.
Given an input vector \( x \), a model function \( f(x) \), and a predicted vector \( y \),
the objective is to identify which components of \( x \) contribute to the generation of the prediction. To accomplish this, the input features are systematically altered by introducing perturbations \( \delta x \)).
This approach allows us to infer the importance of specific features by quantifying the extent to which the predicted vector  \( f(x + \delta x) \).
This approach allows us to infer the importance of specific features by quantifying the extent to which the predicted vector y is affected by the introduction of these perturbations.



Perturbation-based methods are focusing on \("local\) \(interpretability"\), meaning they explain specific predictions rather than providing an overarching understanding of the model’s behavior.
This local explanation is vastly inferior to global interpretability.
Regardless, sometimes it can be extended, to provide broader insights into the models logic.

It works in a similar way to the human visual system and the way we perceive the visual world,
so that by obscuring different parts of an object, we can significantly alter our own eye's perception of the object.
By analysing the effect of perturbations on the model's output, these methods identify the most important features for a given prediction.

As a prominent implementation of this method is LIME (Local Interpretable Model-agnostic Explanations), which will be discussed in a following section section~\ref{subsec:methods-for-model-interpretation}, it was selected
for use in the project due to its simplicity and effectiveness in interpreting the model's predictions.


There are multiple types of perturbations:
\begin{itemize}
    \item \textbf{Noise insertion}: adding some type of noise (Gaussian or random) to parts of the input data
    which can reveal the sensitivity of the model to small variations.
    \item \textbf{Blurring/Pixel Modification}: when working with image data, perturbations can be created by blurring parts of an image or modifying pixel intensities, which helps to understand how the clarity or sharpness of particular regions affects the prediction.
    \item \textbf{Feature Deletion for structured data}: remove or zeroing specific features, determining how significant that feature really is to the prediction.
    \item \textbf{Text Data Perturbations}: for text data, this involves removing words, phrases, characters, but it can also mean swapping tokens with synonyms to gauge their importance to the output.
\end{itemize}

Given the project's domain, I only used perturbations used in image processing tasks, and these exclusively consists of modifying pixels and groups of pixels to determine their usefulness to the detection.
In the following, I will discuss two variations of this standard perturbation based approach.

\paragraph{Game theoretic approach with perturbation}\label{par:game-theoretic-approach-with-pertubation}

 is a model-agnostic interpretation method that leverage cooperative game theory to explain model predictions.
This translates to decomposition of the input image into a number identical, equally-sized components, referred to as \("\)features\("\).
These are processed further and a so called Shapley value is calculated to each feature, based on their contribution to the prediction.
This will be projected onto the image, where each feature will be coloured according to this value, thus facilitating easier reading and comprehension.
A detailed examination of this process can be found in reference \ref{par:shap}, where the operational mechanics of this methodology will be elucidated.

Based on the work of Lundberg~\cite{lundberg2017unifiedapproachinterpretingmodel} and Liang~\cite{LIANG2021168},
these method, namely SHAP (SHapley Additive exPlanations), is particularly useful for image processing tasks, as it provides detailed insights into the model's decision-making process.
On grounds of the aforementioned, I chose to use SHAP in my project, as it offers a comprehensive and theoretically grounded approach to interpreting the model's predictions.

\paragraph{Adversarial-based interpretation}\label{par:adversarial-based-interpretation}

methods, based on the work of Lian~\cite{LIANG2021168},
represent a subtype of perturbation methods,
the objective of which is to provide more robust interpretations.
These methods address a common issue in deep neural networks (DNNs), namely poor generalisation performance, as highlighted in the base of my work~\cite{LIANG2021168}.
Deep neural networks (DNNs) are highly sensitive to perturbations in input data, which can result in significant alterations to the predictions made,
thus affecting the stability and reliability of the interpretations produced.
The earliest adversarial-based interpretation methods, as exemplified by the approaches put forth by Fong et al. (2017) and other researchers,
employed techniques to mitigate the impact of perturbations.
This was achieved by utilising random masks and regularising the masks with total-variation norms to smoothen the images that had been perturbed.

Although these methods enhance robustness, they necessitate manual tuning of hyperparameters and yield interpretations of relatively low resolution,
thereby constraining their capacity for fine-grained analysis.
On these grounds, they are not suitable for applications that require high-resolution interpretations, and such I did not use them in my project.

\subsubsection{Concept-based Interpretation}\label{subsubsec:concept-based-interpretation}

Concept-based interpretation methods, based on the work of Liang~\cite{LIANG2021168}, employ predefined sets of human-understandable concepts to provide explanations for deep learning models.
The generation of these concepts is typically conducted with the assistance of domain experts, thereby rendering this method particularly
useful for interpreting the decision-making processes of models in a manner that aligns with human intuition.
The process commences with the selection of a concept set (\(C\)), which encompasses images or data that correspond to specific attributes of the input.
For instance, the concept set may include images of tiger stripes for an image of a tiger.

The definition of these concepts is facilitated by humans, and the concept set is subsequently incorporated into the deep neural
network (DNN) in order to ascertain which concepts are most relevant to the model's predictions.
Two prominent concept-based interpretation methods are Testing with Concept Activation Vectors (TCAV) and Network Dissection (ND).
TCAV explains the behaviour of the model in question by employing human-defined concepts, namely Concept Activation Vectors (CAVs)
, which quantify the importance of said concepts to the model's predictions through directional derivatives.
This approach allows for the assessment of the sensitivity of the model's predictions to specific concepts,
such as texture or colour, across a range of inputs.

In contrast, Network Dissection quantifies the relationship between internal neural network features and visual concepts,
utilising metrics such as Intersection over Union (IoU) to ascertain the degree to which individual neurons correspond to
concepts such as colour, texture, or objects. This approach facilitates a comprehensive understanding of the specific layers
and neurons in a CNN that are responsible for detecting various concepts, thereby offering insights into the interpretability of each layer.

Based on the work of Liang~\cite{LIANG2021168}, these methods are particularly useful for image processing tasks,
as they provide human-understandable explanations for the model's predictions.
However, they require the input of domain experts to define the concepts, which can be time-consuming and may introduce bias.
Furthermore, the interpretability of these methods is limited by the quality of the concept set, which may not capture all
the relevant features of the input data.

As a result, I found that these methods were not suitable for my project, as they
require a high level of domain expertise and may not provide the level of interpretability required for fine-grained analysis.





\subsubsection{Comparison of chosen Model-Agnostic Methods}\label{subsubsec:comparison-of-model-agnostic-methods}

In this subsection, I will compare two of the most popular model-agnostic interpretation methods, LIME and SHAP.
Based on the work of Liang~\cite{LIANG2021168}, these model-agnostic interpretation methods have their strengths and weaknesses.
Through their example I will illustrate the differences between the two methods.

LIME and SHAP while using
different approaches, there are both using some form of perturbation to explain the model's predictions.
LIME approximates the model locally by fitting a simple interpretable model around
the prediction of interest, providing local explanations by perturbing the input
data and observing the changes in the model's predictions.
It is generally faster and simpler, making it suitable for quick, local explanations.


On the other hand, SHAP is based on cooperative game theory and uses Shapley values to attribute the contribution of
each feature to the prediction.
It provides both local and global explanations by considering all possible combinations of features,
producing consistent and theoretically sound explanations.


Based on the discussions by~\cite{lundberg2017unifiedapproachinterpretingmodel}, Kernel SHAP, a variant of SHAP,
is particularly useful for image processing tasks, as it can handle high-dimensional data efficiently.
It is based on the idea of approximating the model with LIME and using the Shapley values to explain
the model's predictions.
This approach combines the strengths of both LIME and SHAP, providing accurate and efficient explanations for
image processing models.

However, SHAP is more computationally intensive due to the need to consider all possible feature combinations.
During the interpretation scripts run, SHAP had 4 times larger GPU memory usage than LIME, while the runtime was ~100 times longer.
While LIME is flexible and can be applied to any model and data type, SHAP offers a more comprehensive and theoretically
grounded approach at a higher computational cost.

\subsection{Model-Specific Methods}\label{subsec:model-specific-methods}
The base idea behind this category is to ground our interpretation on the models internal state.
By this in the field of image processing, it aims to somehow visualize some parameters of the inner  state of the model, and project it back to the input image.
Multiple methods exist in this category, such as Class Activation Maps, Gradient-based methods.
The following paragraphs, present a discussion of the Class Activation Maps with reference to the work of Bany Muhammad and his introduction of his interpretation method~\cite{Muhammad_2020}.
Its implementation will be discussed in the next section.
Furthermore, I will discuss the Gradient-based methods based on the work of Selvaraju titled \("\)Grad-CAM: Visual
Explanations from Deep Networks via Gradient-Based Localization\("\)~\cite{Selvaraju_2019}.




\subsubsection{Class Activation Maps}\label{subsubsec:CAM}

Class Activation Maps (CAMs) are a model-specific interpretation method designed to highlight the regions or features
in an image that are most pertinent to a given model.
In an image, these are the regions that are most relevant to a model's decision, particularly in the context of classification tasks.
The operation of CAMs is based on the projection of the activations from the  convolutional layers back onto the input image.
This results in the generation of a heatmap, visually represents the areas that contributed
the most to the prediction~\cite{Muhammad_2020}.
This method provides valuable insights into which features the model is focusing on, enabling a better understanding of
the model's decision-making process.

The original CAM technique, first introduced by Zhou et al. (2016),
is specifically designed for models that include global average pooling layers before the final output layer,
often referred to as the \("\)head\("\).
These pooling layers facilitate and are responsible for the smooth projections (feature maps) on the methods output image.
While the CAM technique is highly effective, it is somewhat limited in that it is optimized for this specific architectural
configuration, which restricts its versatility.
It is less flexible for networks that rely on fully connected layers following the convolutional layers~\cite{Zhou_2016}.

To illustrate, in a classification task where a model identifies a vehicle in an image, CAM would generate a heat map over and around the vehicle.
This indicates that the model primarily focused on that object when making its decision, which is a useful insight.
In the absence of this information, there is a possibility that the network may be detecting a feature that is not
exclusive to that particular object class and this can result in greater confusion within the model.

This type of visual feedback is especially valuable in domains such safety critical (automotive, etc.) or medical fields,
where understanding why the model identifies certain patterns is critical~\cite{Muhammad_2020}.

In response to the architectural limitations of the original CAM method, several improvements have been proposed.
One significant extension is Grad-CAM (Gradient-weighted Class Activation Mapping), which eliminates the dependence
on specific model architectures by using gradients to compute the heatmaps.
Details of Grad-CAM will be further discussed in the next section~\cite{Selvaraju_2019}.

\subsubsection{Gradient-based Methods}\label{subsubsec:gradient-based-methods}

Gradient-based methods, such as Grad-CAM (Gradient-weighted Class Activation Mapping),
take a different approach by utilizing the gradients flowing through the backpropagation of
the network. It can be used to localize the features within the image, that the models learn on.
Grad-CAM generates heatmaps that highlight the regions of the image that contributed the most to the model's output.
Selvaraju et al. (2019) demonstrated the effectiveness of this technique in their work titled
\("\)Grad-CAM: Visual Explanations from Deep Networks via Gradient-Based Localization\("\)~\cite{Selvaraju_2019}.

By leveraging gradients, Grad-CAM provides more precise localization of features compared to basic CAM techniques,
making it particularly useful for deep neural networks in image recognition tasks.

At the end, this method, despite of being more accurate than the vanilla EigenCAM, I did not use it in my project due
to two main considerations.

First of all, the method is more computationally expensive, and requires more memory to run.

Secondly, the method is more complex to implement and requires more input from the model's internal state.
In order to ensure the model calculates the gradients correctly, it should be fed with the whole back-propagation chain,
which has made the other interpretation methods more time-consuming to implement and run.

Based on that I chose to use the CAM method in my project, as it is simpler to implement and provides valuable insights into the model's decision-making process.

\newpage
\subsubsection{Comparison of Model-Specific Methods}\label{subsubsec:comparison-of-model-specific-methods}

Both CAM and Grad-CAM methods have been widely adopted for interpreting convolutional neural networks
and are valuable tools for visualizing and understanding model behaviour.
While CAM is limited by its dependence on specific model architectures, Grad-CAM offers a more flexible and
generalizable approach that can be applied to a wide range of models.
The use of gradients in Grad-CAM allows for more precise localization of features within the image,
providing valuable insights into the decision-making process of the model.

In contest of the resulting visualizations, the output of the two networks tends to be really similar to each other.
Which can be because the two methods are based on the same principle, and the implementation of the Grad-CAM
is a direct extension of the CAM method, utilizing the activation maps with the gradients to provide the explanation.

\subsection{Comparison of Interpretation Methods}\label{subsec:evaluation-interpretation-methods}

In this section, I will compare the model-agnostic and model-specific interpretation methods discussed and chosen in the previous subsections
and paragraph.
Model-agnostic and model-specific interpretation methods each offer unique strengths and weaknesses when applied to
an image processing model.
Both approaches aim to provide insights into the model's decision-making.
However, they are fundamentally different in terms of their underlying assumptions, flexibility, and computational requirements.


Model-agnostic methods often operate by perturbing the input data, observing the changes in the model’s predictions,
and deriving explanations based on these observations as discussed previously~\ref{subsubsec:pertubation-based-interpretation}.
This makes them highly flexible and applicable to a wide variety of tasks and models.
However, this flexibility sometimes comes at the cost of interpretability precision, as model-agnostic methods approximate
how a model behaves based on perturbations, which may not capture the complete depth of the model’s decision logic.

Conversely, model-specific methods are designed to align with the internal mechanisms of a specific model architecture,
thereby facilitating the generation of explanations based on the model's inherent operational logic.
These methods are often more precise, as they can directly access the model's components — such as layers,
weights, activation vectors, or gradients and utilize this information to interpret how the model made its decision.
In fields like image processing, model-specific methods, such as Class Activation Maps (CAMs) or Gradient-based approaches,
provide visual insights into the areas of an image that most influenced the model’s predictions.
By visualizing the model’s internal state, model-specific methods offer a more direct way of understanding the model’s behavior,
especially in tasks where identifying critical features or regions of the data is essential.

In comparing these two categories, the fundamental difference lies in their generalization versus precision.
Model-agnostic methods are particularly useful because of their ability to work across different models and domains,
making them versatile tools in scenarios where multiple model types and architectures are used.
However, they  frequently offer a higher-level view of the model’s behavior, which may be less precise than the
insights gained from model-specific methods.
Conversely, model-specific methods provide a more detailed data for understanding of the decision-making process,
directly interacting with the internal components.
This allows for more detailed insights but limits the applicability to certain model types.





