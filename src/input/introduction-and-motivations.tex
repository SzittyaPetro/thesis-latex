%\subsection{Introduction and Motivations}\label{subsec:introduction-and-motivations} % bevezetés és motivációk
%értelmezés

\section{Introduction}\label{sec:introduction}
The accurate detection of objects that are relevant to the movement of vehicles is a critical task in the field of automotive and transportation systems.
Numerous scenarios can arise where different types of object detection is required.
The development of a highly sophisticated automated driving system requires the efficient gathering and processing of data from sensors located in the chassis to create a virtual model of the surrounding world.

%tervezés célja
The primary challenge of developing autonomous vehicles is the development of systems  to create a reliable and accurate perception of the vehicle's environment.
One of the most widely used solutions is the use of camera sensors as they are able to
provide rich visual data that can be processed rapidly and can give the most information by itself.
The main goal of this project was to develop a software capable of fully fulfilling this need by detecting traffic
objects in real-time while staying robust and reliable in various conditions.
A perfect solution for this task are the deep learning models, especially Convolutional Neural Networks (CNNs).

%értelmezés
Nevertheless, the most significant limitation of these models is that they are frequently regarded as black boxes, lacking a straightforward correlation between input and output.
This lack of transparency in how models arrive at conclusions has led to the development of numerous model interpretation techniques,
as evidenced by the work of a number of different researchers such as Yu Liang~\cite{LIANG2021168}.
These techniques vary in their approach, but they are all aiming to facilitate an insight into the decision-making process of the model.
In a unified manner, these methods are called eXplainable Artificial Intelligence methods, XAI for short.


%feladat indokoltsága
The complexity of interpreting models in these domains arises from two key factors: the intricate nature of the data and the sheer size of models used.
The aforementioned complexity makes it challenging to trace the decision-making process, which unfolds across numerous neurons and layers,
presents a significant challenge.
Nevertheless, it is essential for the identification of potential biases, the improvement of model performance, the fostering the trust of users and regulatory bodies.
These factors are particularly important in the context of traffic object detection, where the consequences of model errors can be severe.
This has heightened the importance of XAI solutions, with the aim of ensuring transparency for end users and developers alike.

As previously discussed in relation to other challenges facing the development of AI systems by Arun Das and Paul Rad in their article
\("\)Opportunities and Challenges in the Development of AI Systems\("\) ~\cite{das2020opportunitieschallengesexplainableartificial}
The General Data Protection Regulation (GDPR) now requires that decisions made by automated systems be
explainable to the user.
Although the GDPR does not explicitly mention XAI and mainly focuses on data privacy,
it seems probable that in the future, there will be a greater expectation of algorithmic transparency and clarification of AI systems.
Additionally, there are several ISO standards that are dedicated to the safety of AI systems,
including ISO 5469\cite{ISO5469:2021}, which sets forth the requirements for ensuring the intended functionality of AI systems .
Furthermore, it addresses the necessity for XAI methodologies to be employed in the advancement of AI systems.

%tervezés célja
It is therefore essential to enhance the interpretability of this project's model through the use of the aforementioned
XAI methods such as SHAP, LIME and EigenCAM in order to maintain safety and ethical standards, given the nature of the
application.
Based on that the secondary objective of this project was to try and implement these interpretation methods and examine their
effectiveness in the context of traffic object detection and try to instate a more transparent and reliable model.

\subsection{Refinement of the task}\label{subsec:Refinement-of-the-task} % feladat finomítása


The principal objective of this project is to develop a software solution capable of detecting traffic objects in real time.
This will entail an investigation into the fundamental principles of deep learning models for object detection, with a
particular emphasis on the construction of a model based on the YOLOv8 architectural framework.
The model will be trained on a dataset comprising images of urban environments, each of which has been annotated with
traffic objects.
Following training, the model's performance will be evaluated based on its capacity to accurately detect traffic objects
in real-time scenarios.

In addition to real-time detection, the secondary objective of the project is to enhance the interpretability of the
model through the utilisation of Explainable AI (XAI) methodologies.
The selection and implementation of both model-agnostic and model-specific interpretation techniques will be undertaken,
with a particular focus on understanding their underlying principles and applying them to the traffic detection model.
The efficacy of these interpretation techniques will be evaluated by assessing their capacity to elucidate the
rationale behind the model's predictions.
This will ensure that the system is both accurate and transparent in its decision-making processes.
The objective of this study is to develop a deep learning model for traffic object detection and to apply XAI
methods to enhance the model's interpretability.

The two tasks were approached as discrete entities, and thus their explanations were presented separately.
Initially, the object detection and the neural network's fulfilment of the task and its evaluation were discussed.
Subsequently, the model's interpretation methods and their outcomes were evaluated.

The choice of dataset, model and interpretation methods was made with these goals in mind, with the intention of optimising the model for the task in question, in line with the findings of the literature and previous studies.
\subsection{Structure}\label{subsec:Structure} % felépítés rövid összefoglalása

The main parts of my thesis are as follows:

\begin{itemize}
    \item Object detection component:
    \begin{itemize}
        \item Literature study about Image processing Deep Convolutional Neural Networks
        \item Yolo architecture and general overview on training and infrastructure.
        \item Cityscapes Dataset and data processing and representation.
        \item Evaluation of model performance
    \end{itemize}
    \item Interpretation methods
    \begin{itemize}
        \item Literature review about model-agnostic and model-dependent interpretation methods
        \item Implementation, use and evaluation of EigenCAM, LIME and SHAP\@.
        \item To analyze the results and draw conclusions on the effectiveness of the model and the XAI solutions for traffic object detection.
    \end{itemize}
\end{itemize}

